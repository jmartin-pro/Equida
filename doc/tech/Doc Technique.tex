

\documentclass[report]{BetterDocument}

\newcommand{\bdd}{base de données}

\title{Équida}
\subtitle{Doc Technique}
\who{MARTIN Justine\\
	BOTTON Léa}
\date{2019}
\place{BTS SIO 2 - Jean Rostand Caen - E4}

\graphicspath{
    {include/appli_mobile/}
    {include/appli_web/}
    {include/contexte/}
    {include/core/}
}

\begin{document}

	\pageDeGarde

	\tableDesMatieres

	\chapter{Contexte et présentation}
	\section{Présentation du contexte}
	%TODO Parler du client et de son métier + besoin (cf : Cahier des charges.pdf)

	\section{Choix techno}

		%TODO Disclamer : Pas tuto SpringBoot, gradle, ionic, ... (mettre des liens tutos)

		\subsection{MySql}
		%TODO Pourquoi avoir choisi MySql et pas autre chose

		\subsection{Spring Boot}
		%TODO Pourquoi Spring boot, parler des ""parts du marchés"", avantages, ...

		\subsection{Ionic}
		%TODO Pourquoi Ionic, écrire du code en code Js, multiplateforme, ...

		\subsection{Gradle}
			%On fait une description brève car on en reparle plus en détail dans "Intéraction entre les différentes parties du projet"
			Gradle est un "build automation system" (moteur de production). Il est un équivalent plus récent et complet à Maven. Il possède de meilleure performances, possède un bon support dans de nombreux IDE et permet d'utiliser de nombreux dépots, dont ceux de Maven.

	\section{Organisation du projet}
		\subsection{Git et branches}
			\subsubsection{Branches}
			%TODO Parler de master (=version prod), develop, features/XXX

			\subsubsection{Nommenclature}
			%TODO Parler des nommenclatures (cf : fichier CONVENTIONS.md)

		\subsection{Les différents dossiers}
			\subsubsection{Doc}
			%TODO ((Parler de pk latex?))

			\subsubsection{SQL}
			%TODO Expliquer + nommenclature et table de version

			\subsubsection{Sources}
			%TODO Ionic = ionic, spring = code spring boot(webapp, rest, core) (on en reparlera plus tard dans la doc)

		\subsection{Trello}
		%TODO Fournir le lien et expliquer étiquettes, Listes

	\section{Interactions entre les différentes parties du projet}
		\subsection{Les différentes parties}
			Le projet Équida est composé de 2 applications. Une l'application web, qui est également l'application principale, une application mobile qui est à usage principal des utilisateurs. Les 2 applications s'appuient sur la même \bdd{}. L'application web y est directement connectée. L'application mobile, elle, passe par une API. En effet, si celle ci se connecterait directement à la \bdd{} comme c'est le cas pour l'application web, une personne mal intentionnée serait en mesure de décompiler l'application mobile afin d'obtenir les identifiants de la \bdd{}. L'utilisation de cette API empêche donc notamment ce problème de sécurité.

			\begin{figure}[H]
				\centering\includegraphics[width=0.45\textwidth, keepaspectratio]{res/diag_infra.png}
				\caption{La connexion à la BDD selon le projet}
			\end{figure}

			L'API ainsi que l'application web utilisent sur le Framework Spring Boot. Ces 2 applications font donc parti de 2 projets différents, "webapp" pour la partie web et "rest" pour l'api. Celles si demandant un code identique pour les Services, les Entités ainsi que les Repository, le choix a donc été fait de faire un projet commun dénomé "core" dans lequel on peut retrouver tout le code qui sera commun aux 2 autres parties, non seulement concernant les éléments cités plus haut mais également concernant les exceptions ou certains outils.

			\begin{figure}[H]
				\centering\includegraphics[width=0.75\textwidth, keepaspectratio]{res/diag_projet.png}
				\caption{Les dépendances entre les projets}
			\end{figure}

		\subsection{Configuration Gradle}

			Pour gérer correctement les différents projets basés sur Spring, leur dépendances ainsi que leur configuration nous avons donc utilisé Gradle comme mentionné plus haut. Dans le dossier "src/Spring" on retrouve le "build.gradle" qui se charge de configurer tout le projet. On peut observer la configuration suivante pour tout les projets.

			\begin{figure}[H]
				\centering\includegraphics[width=0.75\textwidth, keepaspectratio]{res/gradle_allprojects.png}
				\caption{Configuration Gradle de tous les projets}
			\end{figure}

			On définit donc la version de Spring à utiliser, en plus des dépendences commune à chaque projet (spring-boot-starter-web, spring-boot-starter-data-jpa, ...). On va par la suite définir les dépendances uniques à chaque projet.

			\begin{figure}[H]
				\centering\includegraphics[width=0.75\textwidth, keepaspectratio]{res/gradle_project.png}
				\caption{Configuration Gradle individuelle des projets}
			\end{figure}

			De même, concernant le projet core, on active uniquement la compilation en jar (comme une lib) et non pas en jar bootable (comme c'est le cas lorsque l'on utilise Spring Boot).

			D'autre scripts "build.gradle" se trouvent dans chaque dossiers du projet, cependant, ceux ci ne configurent que le nom du projet à l'issue du build, la version du JDK utilisée ainsi que le package de base du projet.


	\chapter{Core}
	\section{Organisation des packages}
	\section{Exemple d'Entity}
	\section{Exemple de Repository}
	\section{Exemple de Service}
	\section{Exemple d'exception (NotFoundException)}
	\section{Authentification}
	\section{Utils}


	\chapter{Application web}

	\section{Organisation des packages}

		\subsection{Packages}

			%TODO Léa : Image et décrire le but des packages (src)

			L'organisation des packages se présente comme suit :

			\begin{figure}[H]
				\centering\includegraphics[width=0.33\textwidth, keepaspectratio]{res/package.png}
				\caption{Packages de WebApp}
			\end{figure}

			\begin{description}
				\item[attribute :]{Contient la classe InputOutputAttribute qui gère toutes les variables dont on pourrait avoir besoin}
				\item[config :]{Contient les classes relatives à la configuration et la sécurité de l'application}
				\item[controller :]{Contient toutes les contrôleurs qui héritent de la classe AbstractWebController}
				\item[form :]{Contient tous les formulaires qui héritent de la classe IForm}
				\item[interceptor :]{Contient la classe UserInterceptor qui hérite de HandlerInterceptorAdapter (voir le fonctionnement de Interceptor : \ref{interceptor} et/ou \url{https://www.tutorialspoint.com/spring_boot/spring_boot_interceptor.htm})}
				\item[route :]{Contient toutes les routes qui héritent de IRoute}
			\end{description}

		\subsection{Resources}

			%TODO Léa : Comme avec les packages mais sur les ressources

			L'organisation des ressources se présente comme suit :

			\begin{figure}[H]
				\centering\includegraphics[width=0.33\textwidth, keepaspectratio]{res/ressources.png}
				\caption{Ressources de WebApp}
			\end{figure}

			\begin{description}
				\item[static.assets :]{Contient toutes les ressources statiques qui ne nécessitent aucune compilation}
				\begin{description}
					\item[img :]{Contient toutes les images de l'application}
					\item[js :]{Contient tous les fichiers JavaScript notamment celui pour la gestion des classements à une course d'un cheval et celui pour la page d'accueil avec son carrousel et son menu}
					\item[styles :]{Contient le fichier css de base}
					\item[vendors :]{Contient toutes les dépendances externes du projet soit : Materialize (librairie qui gère le design des vues de l'application), jQuery, Google (police de caractères)}
				\end{description}

				\item[templates :]{Contient tous les fichiers Freemarker}
				\begin{description}
					\item[error :]{Contient les fichiers ftl pour les erreurs 403, 404, 500}
					\item[layouts :]{Contient les fichiers ftl communs à toutes les pages de l'application}
					\item[view :]{Contient toutes les vues de l'applications (lister, consulter, form)}
					\item[login.ftl :]{Fichier utilisé par SpringSecurity pour la page d'authentification}
				\end{description}
			\end{description}

	\section{Parler configuration de l'application}

		\subsection{application.properties}

			%TODO Justine

		\subsection{Configuration par le code}

			%TODO Justine

	\section{Fichiers resources}

		\subsection{FreeMarker}
			FreeMarker est un moteur de template basé sur Java qui est à l'origine de la génération de pages web dynamiques dans une architecture logicielle.\newline
			FreeMarker lit donc les fichiers Model, les combine avec les objets Java pour finalement générer un document de sortie type HTML (dans un format de fichier FTL FreeMarker Template Language).

			\subsubsection{Page de base}

				%TODO Léa : Expliquer le fonctionnement de la page de base et les macros
				La page base.ftl correspond à la page type de l'application. On y retrouve ainsi ce qui sera inclus sur toutes les pages de l'application. \newline
				Ainsi, base.ftl contient le header de la page avec l'inclusion de la feuille de style ; sa partie body contient, elle, le contenu du fichier nav.ftl correspondant au menu, ainsi que le footer et les fichiers de scripts nécessaires.\newline
				Les macros permettent le chargement du contenu aux endroits prévus à cet effet. Par exemple, la macro @content, présente dans le body de base.ftl, chargera le code du fichier x.ftl à cet endroit.

				\begin{figure}[H]
					\centering\includegraphics[width=1\textwidth, keepaspectratio]{res/fonctionnementBaseFtl.png}
					\caption{Exemple du fonctionnement de la page base.ftl avec la page lieux/lister.ftl}
				\end{figure}

				\begin{figure}[H]
					\centering\includegraphics[width=1\textwidth, keepaspectratio]{res/exempleFonctionnementBaseLieuFtl.png}
					\caption{Rendu lors du chargement de la vue qui liste les lieux (correspondant à la page lieux/lister.ftl précédente)}
				\end{figure}

			\subsubsection{Page d'erreur}

				%TODO Léa : Expliquer que les pages sont chargés automatiquement par spring et reprenne design de base
				Les pages d'erreur sont chargées automatiquement par Spring et contiennent des messages explicites. Nous avons gérés les erreurs 403 (permissions non autorisées), 404 (page inexistante) et 500 (exception lors de l'exécution du code). Elles reprennent, elles aussi, le design de base de l'application (base.ftl).

				\begin{figure}[H]
					\centering\includegraphics[width=0.75\textwidth, keepaspectratio]{res/codeErreur404.png}
					\caption{Code de l'erreur 404}
				\end{figure}

				\begin{figure}[H]
					\centering\includegraphics[width=0.85\textwidth, keepaspectratio]{res/exempleErreur404.png}
					\caption{Rendu lors d'une tentative de chargement d'une page inexistante}
				\end{figure}

			\subsubsection{Autre}

				%TODO Justine : Fichiers include + fichiers normaux

	\section{Parler authentification}

		\subsection{Gestion template et controller}

			%TODO Justine

		\label{interceptor} \subsection{Interceptor}

			%TODO Justine

	\section{Exemple Route}

		%TODO Léa : Expliquer les méthodes de l'interface IRoute et donner un exemple de route.
		L'interface IRoute décrit les méthodes qui doivent être implémentées par les classes filles.\newline
		Ainsi chaque fichier route contiendra une méthode getUri(), une méthode getView() et getTitle() qui retourneront respectivement l'URL, la vue et le titre à utiliser dans la page concernée (pour la route correspondante).

		\noindent
		Par exemple, pour PaysRoute, qui est donc la route principale selon notre nomenclature, l'URL correspond à /pays, c'est à cette url là, qu'on chargera la vue pays/lister avec le titre "Les pays".

		\begin{figure}[H]
			\centering\includegraphics[width=0.75\textwidth, keepaspectratio]{res/paysRoute.png}
			\caption{Code de l'interface et utilisation par la classe fille PaysRoute}
		\end{figure}

		\begin{figure}[H]
			\centering\includegraphics[width=0.85\textwidth, keepaspectratio]{res/renduPaysRoute.png}
			\caption{Rendu obtenu avec la vue pays/lister}
		\end{figure}

	\section{Exemple Form}

		%TODO Léa : Expliquer la classe mère IForm et donner un exemple de formulaire (Avec Add et Update et le Neutre)
		La classe mère IForm est une classe abstraite qui utilise la générécité ce qui nous permettra d'adapter les méthodes en fonction de l'entité x pour laquelle le formulaire est fait. L'héritage nous permet donc d'utiliser les variables et méthodes déclarées dans le formulaire neutre xForm.\newline
		Par exemple, prenons l'entité Lieu. On créé le formulaire "neutre" LieuxForm qui héritera de IForm et qui permettra de définir les éléments communs aux formulaire d'ajout et de modification.\newline
		On fera donc hériter de ce formulaire "neutre" LieuxAddForm et LieuxUpdateForm et on passera, dans un premier cas, le la variable isCreation à true et dans le second, à false.

		\begin{figure}[H]
			\centering\includegraphics[width=1\textwidth, keepaspectratio]{res/IForm.png}
			\caption{Exemple implémentation interface IForm avec LieuxForm}
		\end{figure}

		\begin{figure}[H]
			\centering\includegraphics[width=1\textwidth, keepaspectratio]{res/formClassFilles.png}
			\caption{Exemple avec LieuxUpdateForm et LieuxAddForm}
		\end{figure}

	\section{Classe InputOutputAttribute}

		%TODO Justine

	\section{Exemple Controller}

		%TODO Léa : Donner un exemple de controller

		Les différents contrôleurs créés pour les entités x héritent tous de la classe AbstractWebController. \newline
		Les contrôleurs contiennent les différentes fonctions correspondant aux méthodes get, post, patch et delete qui utilisent les services et les routes associées. C'est aussi ici qu'on gère les différentes autorisations afin de savoir qui peut accéder, utiliser telle fonction.

		\noindent
		Par exemple, dans EncheresController, la fonction addGet n'est possible que pour un utilisateur ayant le role 'ADMIN', elle est reliée à l'URL EncheresAddRoute.RAW\_URI (URL stockée dans la variable RAW\_URI de la classe EncheresAddRoute). Elle permet de récupérer le formulaire d'ajout d'une enchère.

		\begin{figure}[H]
			\centering\includegraphics[width=0.75\textwidth, keepaspectratio]{res/enchereController.png}
			\caption{Exemple d'un contrôleur}
		\end{figure}


	\chapter{Application mobile}

	\section{API REST}

		\subsection{Organisation des packages}

			%TODO Léa : Image et décrire le but des packages (src)

		\subsection{Parler configuration de l'application}

			\subsubsection{application.properties}

			L'API Rest possède un fichier application.properties qui permet de configurer certaines parties de l'application.

			\begin{figure}[H]
				\centering\includegraphics[width=0.85\textwidth, keepaspectratio]{res/application-properties.png}
				\caption{Configuration par application.properties}
			\end{figure}

			Dans ce fichier configure les informations relatives à l'application dans sa globalité, comme le port à utiliser, la configuration pour connexion avec la \bdd{} (nom utilisateur, mot de passe, ip, ...).

			\subsubsection{Configuration par le code}

			La configuration de Spring Security est directement faite dans le code source grace à l'utilisation de l'annotation "Configuration".

			\begin{figure}[H]
				\centering\includegraphics[width=0.75\textwidth, keepaspectratio]{res/config-httpsecurity.png}
				\caption{Configuration de Spring Security}
			\end{figure}

			Ici, on exige l'authentification sur toutes les requêtes faites à l'API. Celle ci doit être faite en utilisant Basic Authentification (voir \nameref{subsec:basic_auth}). On y active les CORS et on désactive le csrf.

			\begin{figure}[H]
				\centering\includegraphics[width=0.75\textwidth, keepaspectratio]{res/config-cors.png}
				\caption{Configuration des requêtes CORS}
			\end{figure}

			Afin de permettre l'utilisation des requêtes DELETE et PATCH sur l'API, il a fallu changer la configuration des CORS. Le choix a été fait d'autoriser ces requêtes pour toutes les URL afin de simplifier la chose. Pour un code plus propre, il aurait fallu autoriser uniquement ce dont chaque URL utilisait.

			\begin{figure}[H]
				\centering\includegraphics[width=0.75\textwidth, keepaspectratio]{res/config-authentification.png}
				\caption{Configuration de l'authentification}
			\end{figure}

			Enfin, comme pour WebApp, on définit le fonctionnement de l'authentification. Ainsi, vous pouvez retrouver les explications dans \nameref{par:authentification}.

		\subsection{Système d'authentification sur l'API}
			\label{subsec:basic_auth}

			L'api utilisant Basic Authentification il est nécessaire de fournir à Spring Security une implémentation de ce modèle. C'est le rôle d'AuthenticationEntryPointImpl dont voici le code :

			\begin{figure}[H]
				\centering\includegraphics[width=0.75\textwidth, keepaspectratio]{res/AuthenticationEntryPointImpl.png}
				\caption{Code d'AuthenticationEntryPointImpl}
			\end{figure}

			Le code reste extrèmement simple ici car Spring Security nous offre une couche d'abstraction pour l'implémentation du modèle. Ainsi, le seul élément que l'on définit est le realm.

		\subsection{Exemple Route}

			%TODO Léa : Expliquer les méthodes de l'interface IRoute et donner un exemple de route.

		\subsection{Les Dto}

			\subsubsection{L'interface IDto}

				Afin d'alléger les requêtes SQL à la \bdd{} et la sortie en JSON des classes métiers, le choix à été fait d'utiliser les DTO. Aucun lien vers un autre objet n'est fait, seul son ID est laissé et il faudra ensuite utiliser l'URL adéquate afin de récupérer les informations si celles ci nous intéressent.

				Pour uniformiser nos Dto, une interface IDto a été faite.

				\begin{figure}[H]
					\centering\includegraphics[width=0.75\textwidth, keepaspectratio]{res/idto.png}
					\caption{Code de l'interface IDto}
				\end{figure}

				Celle ci se base donc sur une généricité double, T étant notre classe métier provenant du package "com.equida.common.bdd.entity" et U étant la classe Dto correspondante. Cette interface définit 2 méthodes, "<T, U> U convertToDto(T entity)" permettant de convertir une instance de nos classe métier en un Dto et "T convertToEntity()" permettant de convertir notre Dto en une instance de nos classe métiers.

			\subsubsection{Exemple de Dto}

				%TODO Léa : Donner un exemple

		\subsection{Exemple Controller}

			%TODO Léa : Donner un exemple de controller

	\section{Ionic}

		\subsection{Organisation des packages}

			%TODO Léa : Image et décrire le but des packages (src)

		\subsection{Les pages}

			%TODO Léa : Parler de la nommenclature

			\subsubsection{Lister}

					%TODO Léa : Donner un exemple

			\subsubsection{Consulter}

					%TODO Léa : Donner un exemple

			\subsubsection{Ajouter}

					%TODO Léa : Donner un exemple

			\subsubsection{Modifier}

					%TODO Léa : Donner un exemple

		\subsection{Rest Api}

			Pour faciliter l'execution des différentes requêtes à l'API la classe "rest-api.service" a été conçue.

			\subsubsection{Authentification à l'Api}

				\begin{figure}[H]
					\centering\includegraphics[width=0.75\textwidth, keepaspectratio]{res/ionic-rest-auth.png}
					\caption{Code de l'interface IDto}
				\end{figure}

				Ici, "saveCredentials" permet d'enregistrer les informations relatives à l'authentification du client dans le local storage. Il existe "removeCredentials" qui elle, à l'inverse de saveCredentials, supprime les informations relatives à l'utilisateur connecté. On a ainsi 2 méthodes qui permettent de gérer facilement la connexion et la déconnexion d'un utlisateur.

  				La méthode "checkLogin" permet de d'appeller "/api/login" et d'obtenir les informations relatives à l'utilisateur si les informations fournis sont correctes.

				\begin{figure}[H]
					\centering\includegraphics[width=0.75\textwidth, keepaspectratio]{res/ionic-rest-httpoptions.png}
					\caption{Code de l'interface IDto}
				\end{figure}

				Ces informations sont ensuite réutilisées pour permettre l'authentification à l'API. Pour cela la méthode "getHttpOptions" permet d'obtenir l'entête à founir à chaque appel à l'API. Dans le cas où le mot de passe ou le login est null, on redirige l'utilisateur vers la page de connexion.

			\subsubsection{Execution des différentes méthodes}

				Les appels à l'API utilisant les méthode HTTP GET, POST, PATCH et DELETE, une méthode existe, dans la classe "rest-api.service", pour chacune de ces méthodes HTTP. Voici un exemple avec la méthode HTTP GET.

				\begin{figure}[H]
					\centering\includegraphics[width=0.75\textwidth, keepaspectratio]{res/ionic-rest-execmethod.png}
					\caption{Code de l'interface IDto}
				\end{figure}

				On effectue ici l'appel à l'url fournit en paramètre tout en récupérant les entêtes nécessaire pour l'authentification puis, selon que l'appel ai réussi ou non, on retourne le résultat ou on appel handleError afin de générer un message d'erreur ainsi qu'une exception. Il est important de vérifier si le code d'erreur est 401 car si c'est le cas, l'utilisateur est ou désactivé ou alors a son mot de passe de changé. Si tel est le cas, on le déconnexte et on le redirige vers la page de connexion.

				Les autres méthodes ne change pas tellement si ce n'est que pour POST et PATCH on fournis en plus un objet "data" qui correspond aux informations à fournir à l'API mais aussi que le nom des méthodes appellées sur l'objet http change.

				\begin{figure}[H]
					\centering\includegraphics[width=0.75\textwidth, keepaspectratio]{res/ionic-rest-error.png}
					\caption{Code de l'interface IDto}
				\end{figure}

				La méthode "handleError" affiche les informations relatives à l'erreur dans la console du navigateur et se charge également d'afficher une boite de dialogue expliquant qu'une erreur est surveneue. Dans le cadre du développement, cela permet également de rappeller que plus d'informations sont disponibles dans la console.

			\subsubsection{Exemples}

			Ainsi, l'appel aux différents endpoints de l'API tient en quelques lignes et se retrouve extrèmement simpifié comme on peut le voir sur les méthodes suivantes :

				\begin{figure}[H]
					\centering\includegraphics[width=0.75\textwidth, keepaspectratio]{res/ionic-rest-add-delete-get.png}
					\caption{Code de l'interface IDto}
				\end{figure}

			Le fait de retourner une instance de Promise permet d'utiliser "await" sur la méthode et donc d'attendre la réponse de l'API avant de continuer l'execution du code.

		\subsection{Problèmes connus}

			%TODO Léa : Parler refresh si ajout ou modif

			Actuellement, en cas d'erreur sur les formulaires, une boite de dialogue avec un message expliquant qu' "une erreur est survenue" s'affiche. A terme il faudrait afficher un message sur l'écran expliquant le champs sur lequel l'erreur est survenue ainsi que la raison de celle ci. Cela est actuellement impossible à faire car il faut restructurer la façon donc rest-api.service fonctionne afin d'y permettre, d'une part, une meilleure gestion des erreurs, d'autre partle fait de ne pas afficher la boite de dialogue systématiquement dans le cas où l'API retourne un code d'erreur.


	\chapter{Ressources}

\noindent
Github du projet : \url{https://github.com/justine-martin-study/Equida}\\
Trello : \url{https://trello.com/b/jrKixhpu/equida-spring}


\end{document}

% Aide pour la prise enmain de Latex.
% Les lignes qui commencent par "%" sont des commentaires.

% =========================

% Structurer un document

% Dans l'ordre on retrouve les éléments suivants :

% \part{part name}
% \chapter{chapter name}
% \section{section name}
% \subsection{subsection name}
% \subsubsection{subsubsection name}

% =========================

% Afficher une image

%	\begin{figure}[H]
%		\centering\includegraphics[width=0.75\textwidth, keepaspectratio]{res/NotFoundException.png}
%		\caption{Code de NotFoundException}
%	\end{figure}

% caption permet d'afficher une légende
% width permet de définir la taille en largeur de l'image par rapport à la taille de la feuille (ici 75%)
% res/NotFoundException.png -> Nom de l'image

% =========================

% Faire une liste à points

%	Blablabla, ainsi on peut citer :
%	\begin{itemize}
%		\item{signaler une erreur}
%		\item{mieux gérer les ...}
%	\end{itemize}

% =========================

% Faire une liste descriptive

%	Blablabla, on a donc les éléments suivants : 
%	\begin{description}
%		\item[Service]{Fait l'intermédiaire entre ...}
%		\item[Repository]{Permet de faire ...}
%	\end{description}
