\chapter{Contexte et présentation}
	\section{Présentation du contexte}
	%TODO Léa : Parler du client et de son métier + besoin (cf : Cahier des charges.pdf)

	\section{Choix techno}

		%TODO Léa : Disclamer : Pas tuto SpringBoot, gradle, ionic, ... (mettre des liens tutos)

		\subsection{MySql}
		%TODO Léa : Pourquoi avoir choisi MySql et pas autre chose

		\subsection{Spring Boot}
		%TODO Léa : Pourquoi Spring boot, parler des ""parts du marchés"", avantages, ...

		\subsection{Ionic}
		%TODO Léa : Pourquoi Ionic, écrire du code en code Js, multiplateforme, ...

		\subsection{Gradle}
			%On fait une description brève car on en reparle plus en détail dans "Intéraction entre les différentes parties du projet"
			Gradle est un "build automation system" (moteur de production). Il est un équivalent plus récent et plus complet à Maven. Il possède de meilleures performances, un bon support dans de nombreux IDE et permet d'utiliser de nombreux dépots, dont ceux de Maven.

	\section{Organisation du projet}
		\subsection{Git et branches}
			\subsubsection{Branches}
			%TODO Léa : Parler de master (=version prod), develop, features/XXX

			\subsubsection{Nommenclature}
			%TODO Léa : Parler des nommenclatures (cf : fichier CONVENTIONS.md)

		\subsection{Les différents dossiers}
			\subsubsection{Doc}
			%TODO Léa : ((Parler de pk latex? = opensource, meilleure gestion sur git, compilation directe en pdf, ...)) + autres doc

			\subsubsection{SQL}
			%TODO Léa : Expliquer + nommenclature et table de version

			\subsubsection{Sources}
			%TODO Léa : Ionic = ionic, spring = code spring boot(webapp, rest, core) (on en reparlera plus tard dans la doc)

		\subsection{Trello}
		%TODO Léa : Fournir le lien et expliquer étiquettes, Listes

	\section{Interactions entre les différentes parties du projet}
		\subsection{Les différentes parties}
			Le projet Équida est composé de 2 applications. Une application web, qui est également l'application principale et une application mobile qui est à l'usage principal des clients. Les 2 applications s'appuient sur la même \bdd{}. L'application web y est directement connectée. L'application mobile, elle, passe par une API. En effet, si celle ci se connectait directement à la \bdd{}, comme c'est le cas pour l'application web, une personne mal intentionnée serait en mesure de décompiler l'application mobile et ainsi obtenir les identifiants de la \bdd{}. L'utilisation de cette API empêche donc, notamment, ce problème de sécurité.

			\begin{figure}[H]
				\centering\includegraphics[width=0.45\textwidth, keepaspectratio]{res/diag_infra.png}
				\caption{La connexion à la BDD selon le projet}
			\end{figure}

			L'API ainsi que l'application web utilisent le framework Spring Boot. Ces 2 applications constituent donc 2 projets différents, "webApp" pour la partie web et "rest" pour l'API. Ces dernières nécessitant un code identique pour les Services, les Entity et les Repository, nous avons donc fait le choix de créer un projet commun "core". On y retrouve donc tout le code commun aux 2 autres projets (cités précédemment mais également les exceptions ou certains outils).

			\begin{figure}[H]
				\centering\includegraphics[width=0.75\textwidth, keepaspectratio]{res/diag_projet.png}
				\caption{Les dépendances entre les projets}
			\end{figure}

		\subsection{Configuration Gradle}

			Pour gérer correctement les différents projets basés sur Spring, leurs dépendances ainsi que leurs configurations, nous avons donc utilisé Gradle. Dans le dossier \textit{src/Spring} on retrouve le \textit{build.gradle} qui se charge de configurer la totalité du projet. On peut observer la configuration suivante pour tous.

			\begin{figure}[H]
				\centering\includegraphics[width=0.75\textwidth, keepaspectratio]{res/gradle_allprojects.png}
				\caption{Configuration Gradle commune à tous les projets}
			\end{figure}

			On précise donc la version de Spring à utiliser, en plus des dépendances communes à tous les projets (spring-boot-starter-web, spring-boot-starter-data-jpa, ...). Par la suite, on définit les dépendances uniques à chaque projet.

			\begin{figure}[H]
				\centering\includegraphics[width=0.75\textwidth, keepaspectratio]{res/gradle_project.png}
				\caption{Configuration Gradle propre à chaque projet}
			\end{figure}

			De même, concernant le projet core, on active uniquement la compilation en jar (comme une librairie) et non pas en jar bootable (comme c'est le cas lorsque l'on utilise Spring Boot).

			D'autres scripts "build.gradle" se trouvent dans chaque dossier du projet, cependant, ceux-ci ne configurent que le nom du projet à l'issue du build, la version du JDK utilisée ainsi que le package de base du projet.
