\chapter{Application mobile}

	\section{API REST}

		\subsection{Organisation des packages}

			%TODO Léa : Image et décrire le but des packages (src)

		\subsection{Parler configuration de l'application}

			\subsubsection{application.properties}

				%TODO Justine

			\subsubsection{Configuration par le code}

				%TODO Justine

		\subsection{Parler authentification}

			%TODO Justine

		\subsection{Exemple Route}

			%TODO Léa : Expliquer les méthodes de l'interface IRoute et donner un exemple de route.

		\subsection{Exemple Dto}

			%TODO Justine : Expliquer interface IDto
			%TODO Léa : Donner un exemple

		\subsection{Exemple Controller}

			%TODO Léa : Donner un exemple de controller

	\section{Ionic}

		\subsection{Organisation des packages}

			%TODO Léa : Image et décrire le but des packages (src)

		\subsection{Les pages}

			%TODO Léa : Parler de la nommenclature
				Les différentes pages créées (via la commande "ionic g") suivent la même nomenclature : elles sont nommées par un verbe (souvent : lister, ajouter, consulter, modifier). Ces pages sont donc générées dans un dossier du nom de l'entité concernée (ex : pays, ventes).
				Dans chaque sous-dossier créé par la commande ionic, on retrouvera les mêmes types de fichiers.
				Un fichier x.module.ts qui contient
				Un fichier x.page.html qui contient donc le code html de la page ainsi que les composants ionic.
				Un fichier x.page.scss
				Un fichier page.spec.ts
				Un fichier x.page.ts qui contient toutes les méthodes à éxécuter ainsi que l'initialisation de la page.
				On ne travaillerai que sur les fichiers x.page.ts, x.page.html.

			\subsubsection{Lister}

					%TODO Léa : Donner un exemple
						
			\subsubsection{Consulter}

					%TODO Léa : Donner un exemple

			\subsubsection{Ajouter}

					%TODO Léa : Donner un exemple

			\subsubsection{Modifier}

					%TODO Léa : Donner un exemple

		\subsection{Rest Api}

			%TODO Justine

		\subsection{Authentification à l'Api}

			%TODO Justine

		\subsection{Problèmes connus}

			%TODO Léa : Parler refresh si ajout ou modif
			Cette application mobile comporte des erreurs sur lesquelles nous n'avons pas su intervenir. En effet, après un ajout ou une modification (quelque soit l'entité concernée), lorsque l'on redirige vers la page principale qui liste, les informations ne sont pas mises à jour car la page n'est pas actualisée.

			%TODO Justine : Parler affichage message erreur sur le formulaire
